% Options for packages loaded elsewhere
\PassOptionsToPackage{unicode}{hyperref}
\PassOptionsToPackage{hyphens}{url}
\documentclass[
  11pt,
]{article}
\usepackage{xcolor}
\usepackage[margin=1in]{geometry}
\usepackage{amsmath,amssymb}
\setcounter{secnumdepth}{5}
\usepackage{iftex}
\ifPDFTeX
  \usepackage[T1]{fontenc}
  \usepackage[utf8]{inputenc}
  \usepackage{textcomp} % provide euro and other symbols
\else % if luatex or xetex
  \usepackage{unicode-math} % this also loads fontspec
  \defaultfontfeatures{Scale=MatchLowercase}
  \defaultfontfeatures[\rmfamily]{Ligatures=TeX,Scale=1}
\fi
\usepackage{lmodern}
\ifPDFTeX\else
  % xetex/luatex font selection
\fi
% Use upquote if available, for straight quotes in verbatim environments
\IfFileExists{upquote.sty}{\usepackage{upquote}}{}
\IfFileExists{microtype.sty}{% use microtype if available
  \usepackage[]{microtype}
  \UseMicrotypeSet[protrusion]{basicmath} % disable protrusion for tt fonts
}{}
\makeatletter
\@ifundefined{KOMAClassName}{% if non-KOMA class
  \IfFileExists{parskip.sty}{%
    \usepackage{parskip}
  }{% else
    \setlength{\parindent}{0pt}
    \setlength{\parskip}{6pt plus 2pt minus 1pt}}
}{% if KOMA class
  \KOMAoptions{parskip=half}}
\makeatother
\usepackage{color}
\usepackage{fancyvrb}
\newcommand{\VerbBar}{|}
\newcommand{\VERB}{\Verb[commandchars=\\\{\}]}
\DefineVerbatimEnvironment{Highlighting}{Verbatim}{commandchars=\\\{\}}
% Add ',fontsize=\small' for more characters per line
\usepackage{framed}
\definecolor{shadecolor}{RGB}{248,248,248}
\newenvironment{Shaded}{\begin{snugshade}}{\end{snugshade}}
\newcommand{\AlertTok}[1]{\textcolor[rgb]{0.94,0.16,0.16}{#1}}
\newcommand{\AnnotationTok}[1]{\textcolor[rgb]{0.56,0.35,0.01}{\textbf{\textit{#1}}}}
\newcommand{\AttributeTok}[1]{\textcolor[rgb]{0.13,0.29,0.53}{#1}}
\newcommand{\BaseNTok}[1]{\textcolor[rgb]{0.00,0.00,0.81}{#1}}
\newcommand{\BuiltInTok}[1]{#1}
\newcommand{\CharTok}[1]{\textcolor[rgb]{0.31,0.60,0.02}{#1}}
\newcommand{\CommentTok}[1]{\textcolor[rgb]{0.56,0.35,0.01}{\textit{#1}}}
\newcommand{\CommentVarTok}[1]{\textcolor[rgb]{0.56,0.35,0.01}{\textbf{\textit{#1}}}}
\newcommand{\ConstantTok}[1]{\textcolor[rgb]{0.56,0.35,0.01}{#1}}
\newcommand{\ControlFlowTok}[1]{\textcolor[rgb]{0.13,0.29,0.53}{\textbf{#1}}}
\newcommand{\DataTypeTok}[1]{\textcolor[rgb]{0.13,0.29,0.53}{#1}}
\newcommand{\DecValTok}[1]{\textcolor[rgb]{0.00,0.00,0.81}{#1}}
\newcommand{\DocumentationTok}[1]{\textcolor[rgb]{0.56,0.35,0.01}{\textbf{\textit{#1}}}}
\newcommand{\ErrorTok}[1]{\textcolor[rgb]{0.64,0.00,0.00}{\textbf{#1}}}
\newcommand{\ExtensionTok}[1]{#1}
\newcommand{\FloatTok}[1]{\textcolor[rgb]{0.00,0.00,0.81}{#1}}
\newcommand{\FunctionTok}[1]{\textcolor[rgb]{0.13,0.29,0.53}{\textbf{#1}}}
\newcommand{\ImportTok}[1]{#1}
\newcommand{\InformationTok}[1]{\textcolor[rgb]{0.56,0.35,0.01}{\textbf{\textit{#1}}}}
\newcommand{\KeywordTok}[1]{\textcolor[rgb]{0.13,0.29,0.53}{\textbf{#1}}}
\newcommand{\NormalTok}[1]{#1}
\newcommand{\OperatorTok}[1]{\textcolor[rgb]{0.81,0.36,0.00}{\textbf{#1}}}
\newcommand{\OtherTok}[1]{\textcolor[rgb]{0.56,0.35,0.01}{#1}}
\newcommand{\PreprocessorTok}[1]{\textcolor[rgb]{0.56,0.35,0.01}{\textit{#1}}}
\newcommand{\RegionMarkerTok}[1]{#1}
\newcommand{\SpecialCharTok}[1]{\textcolor[rgb]{0.81,0.36,0.00}{\textbf{#1}}}
\newcommand{\SpecialStringTok}[1]{\textcolor[rgb]{0.31,0.60,0.02}{#1}}
\newcommand{\StringTok}[1]{\textcolor[rgb]{0.31,0.60,0.02}{#1}}
\newcommand{\VariableTok}[1]{\textcolor[rgb]{0.00,0.00,0.00}{#1}}
\newcommand{\VerbatimStringTok}[1]{\textcolor[rgb]{0.31,0.60,0.02}{#1}}
\newcommand{\WarningTok}[1]{\textcolor[rgb]{0.56,0.35,0.01}{\textbf{\textit{#1}}}}
\usepackage{longtable,booktabs,array}
\usepackage{calc} % for calculating minipage widths
% Correct order of tables after \paragraph or \subparagraph
\usepackage{etoolbox}
\makeatletter
\patchcmd\longtable{\par}{\if@noskipsec\mbox{}\fi\par}{}{}
\makeatother
% Allow footnotes in longtable head/foot
\IfFileExists{footnotehyper.sty}{\usepackage{footnotehyper}}{\usepackage{footnote}}
\makesavenoteenv{longtable}
\usepackage{graphicx}
\makeatletter
\newsavebox\pandoc@box
\newcommand*\pandocbounded[1]{% scales image to fit in text height/width
  \sbox\pandoc@box{#1}%
  \Gscale@div\@tempa{\textheight}{\dimexpr\ht\pandoc@box+\dp\pandoc@box\relax}%
  \Gscale@div\@tempb{\linewidth}{\wd\pandoc@box}%
  \ifdim\@tempb\p@<\@tempa\p@\let\@tempa\@tempb\fi% select the smaller of both
  \ifdim\@tempa\p@<\p@\scalebox{\@tempa}{\usebox\pandoc@box}%
  \else\usebox{\pandoc@box}%
  \fi%
}
% Set default figure placement to htbp
\def\fps@figure{htbp}
\makeatother
\setlength{\emergencystretch}{3em} % prevent overfull lines
\providecommand{\tightlist}{%
  \setlength{\itemsep}{0pt}\setlength{\parskip}{0pt}}
\usepackage{bookmark}
\IfFileExists{xurl.sty}{\usepackage{xurl}}{} % add URL line breaks if available
\urlstyle{same}
\hypersetup{
  pdftitle={Surprising Patterns in Global Diabetes and Obesity Trends},
  pdfauthor={Your Name},
  hidelinks,
  pdfcreator={LaTeX via pandoc}}

\title{Surprising Patterns in Global Diabetes and Obesity Trends}
\author{Your Name}
\date{2025-11-10}

\begin{document}
\maketitle

\section{Introduction}\label{introduction}

\subsection{Context}\label{context}

Non-communicable diseases (NCDs) such as diabetes represent a growing
global health challenge. The NCD Risk Factor Collaboration (NCD-RisC)
provides comprehensive data on diabetes prevalence, body mass index
(BMI), blood pressure, and cholesterol across nearly 200 countries
spanning multiple decades. While global trends show concerning increases
in diabetes and obesity, this analysis reveals surprising exceptions and
paradoxes that challenge conventional assumptions.

\subsection{Research Objectives}\label{research-objectives}

This report investigates patterns in NCD risk factors using two cycles
of the CRISP-DM methodology:

\textbf{First Cycle:} Establish baseline understanding of global
diabetes trends and regional variations

\textbf{Second Cycle:} Identify surprising patterns including countries
that have improved despite global trends, statistical outliers that defy
the BMI-diabetes relationship, and unexpected gender differences

\section{First CRISP-DM Cycle}\label{first-crisp-dm-cycle}

\subsection{Business Understanding}\label{business-understanding}

Understanding temporal and geographic patterns in diabetes prevalence is
crucial for public health planning. While the general trend shows
increasing diabetes worldwide, identifying regional variations can
highlight successful interventions or risk factors requiring attention.

\textbf{Research Questions:}

\begin{enumerate}
\def\labelenumi{\arabic{enumi}.}
\tightlist
\item
  How has global diabetes prevalence changed from 1990 to 2024?
\item
  Which regions show the highest diabetes burden in 2020?
\end{enumerate}

\subsection{Data Understanding}\label{data-understanding}

The diabetes dataset contains 13,200 observations across 200 countries,
spanning years 1990 to 2022. Data are age-standardized and stratified by
sex.

\begin{longtable}[]{@{}lrrrrr@{}}
\caption{Diabetes Prevalence Summary Statistics by Sex
(2020)}\tabularnewline
\toprule\noalign{}
Sex & Countries & Mean (\%) & SD & Min (\%) & Max (\%) \\
\midrule\noalign{}
\endfirsthead
\toprule\noalign{}
Sex & Countries & Mean (\%) & SD & Min (\%) & Max (\%) \\
\midrule\noalign{}
\endhead
\bottomrule\noalign{}
\endlastfoot
Men & 200 & 0.13 & 0.06 & 0.03 & 0.34 \\
Women & 200 & 0.14 & 0.07 & 0.02 & 0.33 \\
\end{longtable}

The data show substantial variation in diabetes prevalence across
countries, with values in 2020 ranging from 0.02\% to 0.34\%.

\subsection{Data Preparation}\label{data-preparation}

Data cleaning involved:

\begin{itemize}
\tightlist
\item
  Filtering to include years 1990-2024 for temporal consistency
\item
  Removing observations with missing prevalence values (0 observations
  removed)
\item
  Standardizing country names and creating regional classifications
\item
  Assigning countries to eight major world regions
\end{itemize}

\subsection{Modeling}\label{modeling}

\subsubsection{Global Trends}\label{global-trends}

\begin{figure}

{\centering \includegraphics{../graphs/cycle1_diabetes_trends} 

}

\caption{Diabetes prevalence has increased steadily across all sex categories from 1990 to 2024, with similar trends for males and females.}\label{fig:global-trends}
\end{figure}

Global diabetes prevalence has shown a consistent upward trajectory over
the past three decades. Between 1990 and 2020, mean prevalence increased
from approximately 4.3\% to 9.1\%, representing more than a doubling of
diabetes burden worldwide. The trend is remarkably similar for males and
females, suggesting that rising diabetes rates are not primarily driven
by gender-specific factors.

\subsubsection{Regional Patterns}\label{regional-patterns}

\begin{figure}

{\centering \includegraphics{../graphs/cycle1_regional_comparison} 

}

\caption{Middle East, Oceania, and North America show the highest diabetes prevalence, while African and Asian regions show lower rates.}\label{fig:regional-comparison}
\end{figure}

Regional analysis for 2020 reveals substantial geographic variation. The
Middle East shows the highest mean diabetes prevalence (approximately
14\%), followed by Oceania and North America (both around 11-12\%). In
contrast, Africa and parts of Asia show lower prevalence rates (7-8\%).
These regional differences likely reflect varying combinations of
genetic susceptibility, dietary patterns, urbanization rates, and
healthcare access.

\begin{figure}

{\centering \includegraphics{../graphs/cycle1_top10_countries} 

}

\caption{The highest diabetes prevalence rates are concentrated in Pacific Island nations and Middle Eastern countries.}\label{fig:top-countries}
\end{figure}

\subsection{Evaluation - Cycle 1}\label{evaluation---cycle-1}

\subsubsection{Key Findings}\label{key-findings}

\begin{enumerate}
\def\labelenumi{\arabic{enumi}.}
\tightlist
\item
  \textbf{Universal Upward Trend:} Diabetes prevalence has increased
  globally with no major sex differences in overall trends
\item
  \textbf{Regional Clustering:} High-prevalence regions cluster
  geographically, suggesting shared environmental or cultural risk
  factors
\item
  \textbf{Wide Variation:} The 3-fold variation between regions
  indicates that diabetes is not inevitable, even in high-income
  settings
\end{enumerate}

\subsubsection{Limitations}\label{limitations}

\begin{itemize}
\tightlist
\item
  Age-standardized data masks potential age-specific trends
\item
  Regional classifications may obscure within-region heterogeneity
\item
  Temporal coverage varies by country, with some gaps in earlier years
\end{itemize}

\subsubsection{Questions for Cycle 2}\label{questions-for-cycle-2}

The universal upward trend raises important questions:

\begin{itemize}
\tightlist
\item
  Are there \emph{any} countries that have successfully reduced
  diabetes?
\item
  What drives the regional differences - is it simply BMI, or are other
  factors at play?
\item
  Which countries defy expected patterns based on obesity rates?
\end{itemize}

\section{Second CRISP-DM Cycle}\label{second-crisp-dm-cycle}

\subsection{Refined Business
Understanding}\label{refined-business-understanding}

Based on Cycle 1 findings, this cycle investigates exceptions to the
global trend. Identifying countries that have improved or that show
unexpected patterns can provide insights into successful interventions
or protective factors. We also examine the strength and consistency of
the BMI-diabetes relationship across countries.

\textbf{Research Questions:}

\begin{enumerate}
\def\labelenumi{\arabic{enumi}.}
\setcounter{enumi}{2}
\tightlist
\item
  Which countries have successfully reduced diabetes prevalence despite
  global upward trends?
\item
  How strong is the BMI-diabetes relationship, and which countries are
  statistical outliers?
\item
  Are there unexpected gender patterns in obesity trends?
\end{enumerate}

\subsection{Enhanced Data Preparation}\label{enhanced-data-preparation}

For Cycle 2, we created a merged dataset combining diabetes prevalence
with BMI data, allowing us to examine their relationship. This dataset
contains 13,200 observations where both variables are available for the
same country, year, and sex.

\subsection{Advanced Modeling}\label{advanced-modeling}

\subsubsection{The ``Time Travelers'': Countries That
Improved}\label{the-time-travelers-countries-that-improved}

While most countries experienced rising diabetes rates between 2000 and
2020, \textbf{19 countries} (9.5\%) actually \emph{reduced} their
diabetes prevalence. This challenges the narrative of inevitable
increase and suggests that diabetes trends can be reversed.

\begin{figure}

{\centering \includegraphics{../graphs/cycle2_time_travelers} 

}

\caption{Most dramatic changes in diabetes prevalence (2000-2020). Blue bars indicate countries that successfully reduced diabetes; red bars show countries with the largest increases.}\label{fig:time-travelers-plot}
\end{figure}

The most successful country, \textbf{Nauru}, reduced diabetes prevalence
by \textbf{0 percentage points} over this 20-year period. This
represents a 15.4\% relative decrease from 2000 levels. Conversely,
\textbf{Pakistan} experienced the largest increase of \textbf{0.2
percentage points}.

\begin{longtable}[]{@{}llrrr@{}}
\caption{Top 5 Countries That Reduced Diabetes Prevalence
(2000-2020)}\tabularnewline
\toprule\noalign{}
Country & Region & 2000 (\%) & 2020 (\%) & Change (\%) \\
\midrule\noalign{}
\endfirsthead
\toprule\noalign{}
Country & Region & 2000 (\%) & 2020 (\%) & Change (\%) \\
\midrule\noalign{}
\endhead
\bottomrule\noalign{}
\endlastfoot
Nauru & Other & 0.3 & 0.3 & -0.05 \\
Palau & Other & 0.2 & 0.2 & -0.04 \\
Spain & Western Europe & 0.1 & 0.0 & -0.04 \\
Saudi Arabia & Middle East & 0.3 & 0.2 & -0.04 \\
Tuvalu & Other & 0.2 & 0.1 & -0.03 \\
\end{longtable}

These improvements suggest that policy interventions, healthcare
improvements, or lifestyle changes can successfully reverse diabetes
trends even in the face of global increases.

\subsubsection{The BMI-Diabetes
Relationship}\label{the-bmi-diabetes-relationship}

As expected, there is a strong positive correlation between national
mean BMI and diabetes prevalence (r = 0.516, p \textless{} 0.001).
Countries with higher average BMI tend to have higher diabetes rates.

\begin{figure}

{\centering \includegraphics{../graphs/cycle2_bmi_diabetes_correlation} 

}

\caption{Strong positive correlation between BMI and diabetes prevalence across countries (2020). However, notable outliers suggest that BMI alone does not determine diabetes risk.}\label{fig:bmi-diabetes-plot}
\end{figure}

However, this relationship is not universal. The R² value of 0.267
indicates that BMI explains approximately 26.7\% of the variation in
diabetes prevalence, leaving 73.3\% unexplained by BMI alone.

\subsubsection{The ``Rule Breakers'': Statistical
Outliers}\label{the-rule-breakers-statistical-outliers}

We identified \textbf{11 countries} as statistical outliers
(\textgreater2 standard deviations from predicted values), which do not
follow the expected BMI-diabetes relationship:

\begin{itemize}
\tightlist
\item
  \textbf{11 countries} have \emph{higher} diabetes than their BMI would
  predict
\item
  \textbf{0 countries} have \emph{lower} diabetes than their BMI would
  predict
\end{itemize}

\begin{figure}

{\centering \includegraphics{../graphs/cycle2_outliers} 

}

\caption{Countries labeled as outliers show diabetes rates significantly different from predictions based on BMI. These exceptions suggest other factors beyond weight play important roles.}\label{fig:outliers-plot}
\end{figure}

\textbf{Pakistan} shows the most extreme positive deviation, with
observed diabetes prevalence of 0.3\% compared to a predicted value of
0.1\% based on its BMI of 0.2 kg/m². Conversely, **** has diabetes
prevalence of \% against a predicted \%.

\begin{longtable}[]{@{}
  >{\raggedright\arraybackslash}p{(\linewidth - 10\tabcolsep) * \real{0.3100}}
  >{\raggedright\arraybackslash}p{(\linewidth - 10\tabcolsep) * \real{0.0700}}
  >{\raggedleft\arraybackslash}p{(\linewidth - 10\tabcolsep) * \real{0.0500}}
  >{\raggedleft\arraybackslash}p{(\linewidth - 10\tabcolsep) * \real{0.1300}}
  >{\raggedleft\arraybackslash}p{(\linewidth - 10\tabcolsep) * \real{0.1400}}
  >{\raggedright\arraybackslash}p{(\linewidth - 10\tabcolsep) * \real{0.3000}}@{}}
\caption{Most Extreme Statistical Outliers in BMI-Diabetes
Relationship}\tabularnewline
\toprule\noalign{}
\begin{minipage}[b]{\linewidth}\raggedright
Country
\end{minipage} & \begin{minipage}[b]{\linewidth}\raggedright
Region
\end{minipage} & \begin{minipage}[b]{\linewidth}\raggedleft
BMI
\end{minipage} & \begin{minipage}[b]{\linewidth}\raggedleft
Observed (\%)
\end{minipage} & \begin{minipage}[b]{\linewidth}\raggedleft
Predicted (\%)
\end{minipage} & \begin{minipage}[b]{\linewidth}\raggedright
Type
\end{minipage} \\
\midrule\noalign{}
\endfirsthead
\toprule\noalign{}
\begin{minipage}[b]{\linewidth}\raggedright
Country
\end{minipage} & \begin{minipage}[b]{\linewidth}\raggedright
Region
\end{minipage} & \begin{minipage}[b]{\linewidth}\raggedleft
BMI
\end{minipage} & \begin{minipage}[b]{\linewidth}\raggedleft
Observed (\%)
\end{minipage} & \begin{minipage}[b]{\linewidth}\raggedleft
Predicted (\%)
\end{minipage} & \begin{minipage}[b]{\linewidth}\raggedright
Type
\end{minipage} \\
\midrule\noalign{}
\endhead
\bottomrule\noalign{}
\endlastfoot
Pakistan & Asia & 0.18 & 0.29 & 0.13 & Higher Diabetes Than Expected \\
Federated States of Micronesia & Other & 0.38 & 0.33 & 0.17 & Higher
Diabetes Than Expected \\
Mauritius & Other & 0.13 & 0.23 & 0.11 & Higher Diabetes Than
Expected \\
Libya & Other & 0.27 & 0.27 & 0.15 & Higher Diabetes Than Expected \\
Sri Lanka & Other & 0.06 & 0.22 & 0.10 & Higher Diabetes Than
Expected \\
Haiti & Other & 0.06 & 0.22 & 0.10 & Higher Diabetes Than Expected \\
\end{longtable}

These outliers suggest that genetic factors, dietary composition,
physical activity patterns, or healthcare quality may modify diabetes
risk independently of BMI.

\subsubsection{Gender Differences in
BMI}\label{gender-differences-in-bmi}

While diabetes trends are similar for males and females, BMI patterns
show interesting gender differences.

In \textbf{116} countries (58\% of countries analyzed), women have
higher mean BMI than men. This pattern is most pronounced in
\textbf{Japan}, where women's BMI exceeds men's by \textbf{0.1 kg/m²}.

\begin{figure}

{\centering \includegraphics{../graphs/cycle2_gender_flip} 

}

\caption{Countries where women have significantly higher BMI than men, contrary to the global pattern where men typically have higher BMI.}\label{fig:gender-plot}
\end{figure}

\begin{longtable}[]{@{}lrr@{}}
\caption{Regional Patterns in Gender BMI Differences}\tabularnewline
\toprule\noalign{}
Region & Countries & Mean Difference (F-M) \\
\midrule\noalign{}
\endfirsthead
\toprule\noalign{}
Region & Countries & Mean Difference (F-M) \\
\midrule\noalign{}
\endhead
\bottomrule\noalign{}
\endlastfoot
Asia & 7 & 0.04 \\
Eastern Europe & 8 & 0.03 \\
Oceania & 3 & 0.02 \\
Other & 67 & 0.02 \\
Western Europe & 15 & 0.02 \\
Middle East & 5 & 0.01 \\
North America & 3 & 0.01 \\
South America & 7 & 0.01 \\
Africa & 1 & 0.00 \\
\end{longtable}

This gender pattern clusters strongly by region, with Asia showing the
highest concentration. Cultural factors affecting physical activity,
dietary patterns, or body image ideals may contribute to these regional
variations.

\subsection{Final Evaluation}\label{final-evaluation}

\subsubsection{Major Findings}\label{major-findings}

\begin{enumerate}
\def\labelenumi{\arabic{enumi}.}
\item
  \textbf{Diabetes Is Not Inevitable:} 19 countries successfully reduced
  diabetes prevalence between 2000 and 2020, demonstrating that upward
  trends can be reversed through effective interventions.
\item
  \textbf{BMI Explains Much But Not All:} While BMI is strongly
  correlated with diabetes (r = 0.516), 11 countries show diabetes rates
  significantly different from predictions, suggesting important roles
  for genetics, diet quality, physical activity, or healthcare access.
\item
  \textbf{Gender Patterns Vary Geographically:} The conventional pattern
  of higher male BMI is reversed in 58\% of countries, with strong
  regional clustering suggesting cultural influences.
\item
  \textbf{Regional Factors Matter:} Both improvements in diabetes and
  outliers in the BMI-diabetes relationship show geographic clustering,
  indicating that shared environmental, cultural, or policy factors
  influence diabetes risk beyond individual-level factors.
\end{enumerate}

\subsubsection{Implications for Public
Health}\label{implications-for-public-health}

\begin{itemize}
\tightlist
\item
  \textbf{Learn from Success Stories:} Countries that reduced diabetes
  should be studied to identify transferable interventions
\item
  \textbf{Look Beyond Weight Loss:} The existence of low-diabetes
  outliers with elevated BMI suggests that metabolic health can be
  improved through factors beyond weight reduction alone
\item
  \textbf{Consider Cultural Context:} Gender differences in obesity and
  diabetes patterns suggest that interventions may need cultural
  tailoring
\item
  \textbf{Target High-Risk Regions:} Countries showing ``perfect storm''
  patterns of multiple risk factors require urgent, comprehensive
  interventions
\end{itemize}

\subsubsection{Study Limitations}\label{study-limitations}

\begin{enumerate}
\def\labelenumi{\arabic{enumi}.}
\tightlist
\item
  \textbf{Ecological Fallacy:} Country-level correlations do not
  necessarily reflect individual-level relationships between BMI and
  diabetes
\item
  \textbf{Confounding Factors:} Many variables not examined here (diet
  quality, physical activity, healthcare access, medications, genetic
  factors) could explain observed patterns
\item
  \textbf{Data Quality Variation:} Measurement methods and data quality
  vary across countries and time periods
\item
  \textbf{Causation vs.~Correlation:} Our observational data cannot
  establish causal relationships
\item
  \textbf{Time Lag Effects:} BMI and diabetes may have complex temporal
  relationships not captured in cross-sectional comparisons
\end{enumerate}

\section{Conclusions}\label{conclusions}

This analysis of NCD-RisC data reveals that despite concerning global
trends in diabetes prevalence, important exceptions and paradoxes exist.
Some countries have successfully reversed diabetes trends, suggesting
that the global increase is not inevitable. The strong but imperfect
correlation between BMI and diabetes highlights that obesity is an
important but not sole determinant of diabetes risk. Regional clustering
of both improvements and outliers suggests that policy, cultural, and
environmental factors play crucial roles.

Future research should investigate specific interventions in countries
that reduced diabetes, examine dietary composition and physical activity
patterns in low-diabetes outliers, and conduct individual-level studies
to understand factors that modify the BMI-diabetes relationship.

\subsection{Methodological Notes}\label{methodological-notes}

This analysis followed the CRISP-DM (Cross-Industry Standard Process for
Data Mining) methodology across two iterative cycles:

\begin{itemize}
\tightlist
\item
  \textbf{Cycle 1} established baseline understanding of global diabetes
  trends and regional patterns
\item
  \textbf{Cycle 2} identified surprising exceptions and investigated the
  BMI-diabetes relationship
\end{itemize}

All analyses were conducted in R using reproducible methods. Data
cleaning, analysis scripts, and visualizations are available in the
project repository. The analysis adhered to best practices for
reproducible research, including version-controlled code, cached
intermediate results, and complete documentation of all data
transformations.

\section{References}\label{references}

NCD Risk Factor Collaboration (NCD-RisC). (2024). Worldwide trends in
diabetes since 1990. \emph{The Lancet}.

NCD Risk Factor Collaboration (NCD-RisC). (2024). Worldwide trends in
body-mass index. \emph{The Lancet}.

NCD Risk Factor Collaboration (NCD-RisC). (2020). Worldwide trends in
cholesterol. \emph{Nature}.

\newpage

\section{Appendix: Technical Details}\label{appendix-technical-details}

\subsection{Data Sources}\label{data-sources}

\begin{itemize}
\tightlist
\item
  \textbf{Diabetes Data:}
  NCD\_RisC\_Lancet\_2024\_Diabetes\_age\_standardised\_countries.csv
\item
  \textbf{BMI Adult Data:}
  NCD\_RisC\_Lancet\_2024\_BMI\_age\_standardised\_country.csv
\item
  \textbf{BMI Child Data:}
  NCD\_RisC\_Lancet\_2024\_BMI\_child\_adolescent\_country\_ageStd.csv
\item
  \textbf{Cholesterol Data:}
  NCD\_RisC\_Nature\_2020\_Cholesterol\_age\_standardised\_countries.csv
\item
  \textbf{Blood Pressure Data:}
  NCD\_RisC\_Lancet\_2017\_BP\_age\_standardised\_countries.csv
\end{itemize}

\subsection{Software Environment}\label{software-environment}

\begin{Shaded}
\begin{Highlighting}[]
\FunctionTok{sessionInfo}\NormalTok{()}
\end{Highlighting}
\end{Shaded}

\begin{verbatim}
## R version 4.5.1 (2025-06-13 ucrt)
## Platform: x86_64-w64-mingw32/x64
## Running under: Windows 11 x64 (build 26100)
## 
## Matrix products: default
##   LAPACK version 3.12.1
## 
## locale:
## [1] LC_COLLATE=English_United States.utf8 
## [2] LC_CTYPE=English_United States.utf8   
## [3] LC_MONETARY=English_United States.utf8
## [4] LC_NUMERIC=C                          
## [5] LC_TIME=English_United States.utf8    
## 
## time zone: Europe/London
## tzcode source: internal
## 
## attached base packages:
## [1] stats     graphics  grDevices utils     datasets  methods   base     
## 
## other attached packages:
## [1] rmarkdown_2.30         knitr_1.50             ggplot2_4.0.0         
## [4] tidyr_1.3.1            readr_2.1.5            dplyr_1.1.4           
## [7] ProjectTemplate_0.11.1 tibble_3.3.0           digest_0.6.37         
## 
## loaded via a namespace (and not attached):
##  [1] Matrix_1.7-4       bit_4.6.0          gtable_0.3.6      
##  [4] compiler_4.5.1     crayon_1.5.3       tinytex_0.57      
##  [7] tidyselect_1.2.1   parallel_4.5.1     splines_4.5.1     
## [10] scales_1.4.0       yaml_2.3.10        fastmap_1.2.0     
## [13] lattice_0.22-7     R6_2.6.1           labeling_0.4.3    
## [16] generics_0.1.4     pillar_1.11.1      RColorBrewer_1.1-3
## [19] tzdb_0.5.0         rlang_1.1.6        utf8_1.2.6        
## [22] xfun_0.54          S7_0.2.0           bit64_4.6.0-1     
## [25] cli_3.6.5          mgcv_1.9-3         withr_3.0.2       
## [28] magrittr_2.0.4     grid_4.5.1         rstudioapi_0.17.1 
## [31] vroom_1.6.6        hms_1.1.4          nlme_3.1-168      
## [34] lifecycle_1.0.4    vctrs_0.6.5        evaluate_1.0.5    
## [37] glue_1.8.0         farver_2.1.2       rsconnect_1.6.1   
## [40] purrr_1.2.0        htmltools_0.5.8.1  tools_4.5.1       
## [43] pkgconfig_2.0.3
\end{verbatim}

\subsection{Reproducibility}\label{reproducibility}

All analyses can be reproduced by:

\begin{enumerate}
\def\labelenumi{\arabic{enumi}.}
\tightlist
\item
  Loading the project:
  \texttt{setwd("C:/Users/DELL/Documents/NCD\_RisC\_Analysis\_K")}
\item
  Running munging scripts: \texttt{source("munge/01-clean-data.R")}
\item
  Running analysis scripts: \texttt{source("src/cycle1\_analysis.R")}
  and \texttt{source("src/cycle2\_interesting\_insights.R")}
\item
  Rendering this report:
  \texttt{rmarkdown::render("reports/analysis\_report.Rmd")}
\end{enumerate}

\end{document}
